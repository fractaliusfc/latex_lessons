\documentclass{article}
\usepackage[spanish]{babel}
\usepackage[utf8]{inputenc}
\usepackage{amsmath}
\usepackage{amsfonts}

\begin{document}

Esto es una texto de prueba. Esto es una ecuación en la misma linea $x=0$. Esto es otro texto de prueba. Esto es otra ecuación: $vel inicial = 5$.

Símbolos especiales, i) exponente/superíndice: $x^{2}$, ii) subíndice: $x_{0}$, iii) raíz cuadrada: $\sqrt{x}$, iv) funciones: $\log(x)$, $\cos(x)$, $\tan(x)$, $\sin(x)$, $\exp(x)$, $e^{x}$, $\ln(x)$, $\log_{2}(x)$, $\sin^{2}(x)$,
v) fracciones: $\frac{1}{2}$, $\dfrac{1}{2}$, $1/2$.

El otro modo para escribir ecuaciones es el modo \emph{display}. Por ejemplo:

\[x^2\]

Ahora escribamos una integral definida, primero agreguemos el símbolo de la integral junto con la variable:

\[\int x\]

Agreguemos la diferencial:

\[\int x dx\]

Agreguemos los límites de integración:

\[\int_{0}^{2} x dx\]

Cambiemos los límites de integración

\[\int_{x_{0}}^{x_{1}} x dx\]

Cambiemos los límites de integración de nuevo:

\[\int_{x_{2}}^{x_{3}} x dx\]

Hagamos diferentes sumas pero ahora con ecuaciones numeradas:

\begin{equation}
    \sum \frac{1}{n}
\end{equation}

\begin{equation}
    \sum \frac{1}{n^2}
\end{equation}

\begin{equation}
    \sum \frac{n+1}{n^2}
\end{equation}

\begin{equation*}
    \sum_{n=0}^{\infty} \frac{n+1}{n^2}
\end{equation*}

Escribamos unos cuantos límites:

\begin{equation}
\lim_{x \to 0} x^{2} = 0
\end{equation}

\begin{equation}
\lim_{x^{-} \to 0} x^{2} = 0
\end{equation}

\begin{equation}
\lim_{x^{+} \to 0} x^{2} = 0
\end{equation}

Escribamos la ecuación para obtener las soluciones de un polinomio de segundo grado:

\begin{equation}
    x = \dfrac{-b\pm\sqrt{b^{2}-4ac}}{2a}
\end{equation}

Podemos poner palabras en modo matemático con el comando \emph{\textbackslash mathrm}:

\begin{equation}
    v_{\mathrm{ini}} = 0
\end{equation}

\begin{equation}
    \mathrm{H}_{2}\mathrm{O}
\end{equation}

Igualmente podemos usar el modo \emph{inline} para solo poner los símbolos matemáticos que necesitemos: H$_{2}$O

Escribamos otra integral usando letras griegas, primero pongamos el símbolo de integración:

\begin{equation}
    \int
\end{equation}

Ahora pongamos los límites de la integral:

\begin{equation}
    \int_{-\infty}^{\infty}
\end{equation}

Agreguemos el número $e$ de la función exponencial:

\begin{equation}
    \int_{-\infty}^{\infty}e
\end{equation}

Agreguemos el $e^{-x}$:

\begin{equation}
    \int_{-\infty}^{\infty}e^{-x}
\end{equation}

Agreguemos un exponente al $-x$:

\begin{equation}
    \int_{-\infty}^{\infty}e^{-x^{2}}
\end{equation}

Escribimos el diferencial:

\begin{equation}
    \int_{-\infty}^{\infty}e^{-x^{2}} dx
\end{equation}

Agreguemos en número $\pi$: 

\begin{equation}
    \int_{-\infty}^{\infty}e^{-x^{2}} dx = \pi
\end{equation}

Y por último agreguemos una raíz cuadrada:

\begin{equation}
    \int_{-\infty}^{\infty}e^{-x^{2}} dx = \sqrt{\pi}
\end{equation}

También podemos escribir otras letras griegas, minúsculas y mayúsculas: $\alpha$, $\delta$, $\Delta$, $\eta$, $\mu$, $\theta$, $\omega$, $\Omega$, $\psi$, $\Psi$, $\phi$, $\Phi$

Y una buena cantidad de símbolos matemáticos:

\begin{equation}
    x \approx 1
\end{equation}

\begin{equation}
    x \sim 1 \Re
\end{equation}

\begin{equation}
    \vec{x} =(1,1,1)
\end{equation}

Esto es un texto y esto una ecuación inline: $x^{2}=0$
y esto es una ecuación en modo display:
\begin{equation}
    x^{2}=0
\end{equation}

Esto es un texto y esto una ecuación inline: $\frac{n+1}{n^2}$
y esto es una ecuación en modo display:
\begin{equation}
    \frac{n+1}{n^2}
\end{equation}

Esto es un texto y esto una ecuación inline: $\dfrac{n+1}{n^2}$
y esto es una ecuación en modo display:
\begin{equation}
    \dfrac{n+1}{n^2}
\end{equation}

\end{document}
